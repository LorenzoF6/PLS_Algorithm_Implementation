\documentclass{beamer}
\usetheme{AnnArbor}
\usecolortheme{beaver}
\usepackage{amsmath}
\usepackage{ragged2e}
\usepackage{listings}
\usepackage{caption}
\usepackage{subcaption}
\usepackage{graphicx}
\usepackage{fancyvrb}
\usepackage{pdfpages}
\usepackage{siunitx}
\usepackage{multirow}
\usepackage{adjustbox}
\justifying 

\setbeamercolor{block title}{fg=black, bg=green!50!black}
\setbeamercolor{block body}{fg=green!50!black, bg=green!50!black!30!white}
\setbeamercolor{block title alerted}{fg=black, bg=orange!50!white}
\setbeamercolor{block body alerted}{fg=orange, bg=orange!30!white}
\setbeamercolor{block title example}{fg=black, bg=blue!50!white}
\setbeamercolor{block body example}{fg=blue, bg=blue!30!white}

\title[Implementation of the PLS algorithm]{Implementation in MATLAB of the Partial Least Squares algorithm for classification}
\subtitle{Case study: fault detection and diagnosis on steel plates}
\author[L. Ferrari, L. Leoni]{Lorenzo Ferrari \and Lorenzo Leoni}
\institute[University of Bergamo]{ Department of Engineering and Applied Sciences, University of Bergamo}
\date{\today}

\begin{document}
	
	\begin{frame}
		\titlepage
	\end{frame}

	\begin{frame}
		\frametitle{Outline}
		\tableofcontents
	\end{frame}
	
	\section{Introduction}

\begin{frame}
	\frametitle{Introduction to the PLS technique}
	\textbf{Partial least squares} (\textbf{PLS}), as known as \textbf{projection to latent structures}, is a dimensionality reduction technique for maximizing the \textbf{covariance} between the predictor (independent) matrix $X \in \mathbb{R}^{n \times m}$ and the predicted (dependent) matrix $Y \in \mathbb{R}^{n \times p}$ for each component of the reduced space $\mathbb{R}^\alpha$ with $\alpha \le m$, where:
	\begin{itemize}
		\item $n$ = number of observations;
		\item $m$ = number of covariates (input variables);
		\item $p$ = number of dependent variables (output variables);
		\item $\alpha$ = dimension of the reduced space in which $X$ is projected.
	\end{itemize}
\end{frame}

\begin{frame}
	\frametitle{Popular application of PLS}
	This technique is often used in \textbf{fault detection} and \textbf{isolation}. With PLS is possible to treat both regression and classification problems. The matrix $X$ always contains the process variables (e.g. diameter and thickness of a gasket), while the matrix $Y$ only (quantitative) quality variables (e.g. its mechanical seal) in the regression case, whereas in pattern classification the predicted variables are dummy variables ($1$ or $0$) such as:
	\begin{equation}
		Y = 
		\begin{bmatrix}
			1 & \dots & 1 & 0 & \dots & 0 & 0 & \dots & 0\\
			0 & \dots & 0 & 1 & \dots & 1 & 0 & \dots & 0\\
			0 & \dots & 0 & 0 & \dots & 0 & 1 & \dots & 1\\
		\end{bmatrix}^\top
	\end{equation}
	where each column of $Y$ corresponds to a fault class. The first $n_j$ elements of column $j$ are filled with a $1$, which indicates that the first $n_j$ rows of $X$ are data from fault $j$. In this case PLS is called \textbf{discriminant}.
\end{frame}
	\section{Description of the PLS  algorithm}

\begin{frame}[fragile]
	\frametitle{NIPALS algorithm}
	The most popular algorithm used in PLS to compute the model parameters is known as \textbf{non-iterative partial least squares} (\textbf{NIPALS}). There are two versions of this technique:
	\begin{itemize}
		\item \textbf{PLS1}: each of the \textit{p} predicted variables in modeled separately, resulting in one model for each class;
		\item \textbf{PLS2}: all predicted variables are modeled simultaneously.
	\end{itemize}
	The first algorithm is more accurate than the other, however it requires more computational time than PLS2 to find the $\alpha$ eigenvectors into which project the \textit{m} covariates. 
\end{frame}

\begin{frame}
	\frametitle{Data structures}
	Before starting with the description of the algorithm, we recall that:
	\begin{itemize}
		\item the matrix $X \in \mathbb{R}^{n\times m}$ is decomposed into a \textbf{score matrix} $T \in \mathbb{R}^{n\times\alpha}$ and a \textbf{loading matrix} $P \in \mathbb{R}^{m\times\alpha}$ such that $X = \hat{X} + E = T\cdot P^\top + E$, where $E \in \mathbb{R}^{n\times m}$ is the (true) \textbf{residual} matrix for $X$;
		\item the matrix $Y \in \mathbb{R}^{n\times p}$ is decomposed into a \textbf{score matrix} $U\in\mathbb{R}^{n\times\alpha}$ and a \textbf{loading matrix} $Q\in \mathbb{R}^{p\times\alpha}$ such that $Y = \hat{Y} + \widetilde{F} = U\cdot Q^\top + \widetilde{F}$, where $\widetilde{F}\in \mathbb{R}^{n\times p}$ is the (true) \textbf{residual matrix} for $Y$.
		\item the matrix $B\in \mathbb{R}^{\alpha\times\alpha}$ is the \textbf{diagonal regression matrix} such that $\hat{U} = T\cdot B$.
	\end{itemize}
	Therefore:
	\begin{center}
		$Y = \hat{U}\cdot Q^\top + F =  T\cdot B\cdot Q^\top + F$
	\end{center}
	where $F$ is the \textbf{prediction error matrix}; $B$ is selected such that the induced $2$-norm of $F$ is minimized. 
\end{frame}

\begin{frame}[fragile]
	\frametitle{MATLAB code}
	The following MATLAB code implements the PLS2 algorithm:
	\begin{Verbatim}[tabsize=4, commandchars=\\\{\}, frame=topline]
E = X; \textcolor{green}{% residual matrix for X}
F = Y; \textcolor{green}{% residual matrix for Y}
[~, idx] = max(sum(Y.*Y));
\textcolor{green}{% search of the j-th eigenvector}
\textcolor{blue}{for} j = 1:alpha
	u = F(:, idx);
	tOld = 0;
	\textcolor{blue}{for} i = 1:maxIter
		w = (E'*u)/norm(E'*u); \textcolor{green}{% support vector}
		t = E*w; \textcolor{green}{% j-th column of the score matrix for X}
		q = (F'*t)/norm(F'*t); \textcolor{green}{% j-th column of the...}
			\textcolor{green}{% loading matrix for Y}
		u = F*q; \textcolor{green}{% j-th column of the score matrix for Y}
	\end{Verbatim}
\end{frame}

\begin{frame}[fragile]
	\begin{Verbatim}[tabsize=4, commandchars=\\\{\}]
		\textcolor{blue}{if} abs(tOld - t) < exitTol
			\textcolor{blue}{break};
		\textcolor{blue}{else}
			tOld = t;
		\textcolor{blue}{end}
	\textcolor{blue}{end}
	p = (E'*t)/(t'*t); \textcolor{green}{% j-th column of the...}
		\textcolor{green}{% loading matrix of X}
	\textcolor{green}{% scaling}
	t = t*norm(p);
	w = w*norm(p);
	p = p/norm(p);
	\textcolor{green}{% calculation of b and the error matrices}
	b = (u'*t)/(t'*t); \textcolor{green}{% j-th column of the...}
	    \textcolor{green}{% coefficient regression matrix}
	E = E - t*p';  \textcolor{green}{% update of the residuals for matrix X}
	F = F - b*t*q'; \textcolor{green}{% update of the residuals for matrix Y}
	\end{Verbatim}
\end{frame}

\begin{frame}[fragile]
	\begin{Verbatim}[tabsize=4, commandchars=\\\{\}, frame=bottomline]
	\textcolor{green}{% calculation of W, P, T and B2}
	W(:, j) = w;
	P(:, j) = p;
	T(:, j) = t;
	B2 = W*(P'*W)^-1*(T'*T)^-1*T'*Y;
\textcolor{blue}{end}
Y_hat = X*B2; \textcolor{green}{% computation of predictions}
	\end{Verbatim}
For each row of \verb|Y_hat| the fault class is chosen by assigning $1$ to the column whose value si greater than that of the others, $0$ otherwise. \\Moreover, to increase the performances of PLS it is necessary \textbf{normalize} both $X$ and $Y$ before running the algorithm.
\end{frame}

	\section{Fault detection and isolation on steel plates}

\begin{frame}
	TO DO
\end{frame}
	\section{Conclusion}

\begin{frame}
	\frametitle{Comparison between PLS and PCA}
	\begin{table}
		\centering
		\renewcommand\arraystretch{1.3}
		\begin{tabular}{c|c|c}
			\hline
			& \textbf{PLS} & \textbf{PCA} \\
			\hline
			\textit{Type of technique} & Supervised & Unsupervised \\
			\multirow{2}{3cm}{\centering \textit{Goal}} & \multirow{2}{3cm}{\centering Regression and classification} & \multirow{2}{3cm}{\centering Feature reduction for clustering}\\
			& & \\
			\multirow{2}{3cm}{\centering \textit{Aim of the maximization}} & \multirow{2}{3cm}{\centering Covariance between $X$ and $Y$} & \multirow{2}{3cm}{\centering Variance of $X$}\\ 
			& & \\
			\multirow{2}{3cm}{\centering \textit{Eigenvectors orthogonality}} & \multirow{2}{3cm}{\centering Not} & \multirow{2}{3cm}{\centering Yes}\\ 
			& & \\
			\multirow{2}{3cm}{\centering \textit{Type of decomposition}} & \multirow{2}{3cm}{\centering NIPALS (iterative approach)} & \multirow{2}{3cm}{\centering SVD}\\ 
			& & \\
			\hline
		\end{tabular}
	\end{table}
\end{frame}

\begin{frame}
	\begin{figure}
		\centering
		\begin{subfigure}[b]{0.60\textwidth}
			\includegraphics[width=\textwidth]{Images/PLS_vs_PCA.pdf}
		\end{subfigure}
	\end{figure}
	For the central orders PCA is slightly more accurate than PLS in the reconstruction of the matrix $X$ starting from the reduced domain. For the extreme orders, instead, the two techniques are similar.
\end{frame}

\begin{frame}
	\frametitle{Bibliography}
	\begin{block}{Book}
		\textbf{Fault Detection and Diagnosis in Industrial Systems}\\
		L. H. Chiang, E. L. Russel and R. D. Braatz
	\end{block}
		\begin{alertblock}{Dataset}
		\textbf{Faulty Steel Plates}\\
		\small{\url{https://www.kaggle.com/datasets/uciml/faulty-steel-plates}}
	\end{alertblock}
	\begin{exampleblock}{Repository GitHub}
		\small{\url{https://github.com/LorenzoF6/PLS_Algorithm_Implementation.git}}
	\end{exampleblock}
\end{frame}
	
\end{document}